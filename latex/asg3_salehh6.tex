\documentclass[11pt,fleqn]{article}

\setlength {\topmargin} {-.15in}
\setlength {\textheight} {8.6in}

\usepackage{amsmath}
\usepackage{amssymb}
\usepackage{amsthm}
\usepackage{color}
\usepackage[utf8]{inputenc}
\usepackage{listings}
\usepackage{fullpage}
\usepackage{fancyvrb}


\renewcommand{\labelenumii}{\theenumii.}

\newcommand{\mname}[1]{\mbox{\sf #1}}
\newcommand{\pnote}[1]{{\langle \text{#1} \rangle}}
\newcommand\todo[1]{\textcolor{red}{[TODO: #1]}}


\begin{document}

\begin{center}

  {\large \textbf{COMPSCI/SFWRENG 2C03}}\\[2mm]
  {\large \textbf{Data Structures and Algorithms}}\\[2mm]
  {\large \textbf{Ryszard Janicki}}\\[2mm]
  {\large \textbf{McMaster University}}\\[6mm]
  {\huge \textbf{Assignment 3}}\\[6mm]
  {\large \textbf{Name: Hishmat Salehi}}\\[2mm]
  {\large \textbf{MacId: Salehh6}}\\[2mm]
  {\large \textbf{Student number: 400172262}}\\[2mm]


\end{center}

\medskip

\begin{enumerate}
	\item 	
		\begin{enumerate}
			\item 
0: 5 $\rightarrow$ 2 $\rightarrow$ 6 \\
1: 4 $\rightarrow$ 8 $\rightarrow$ 11 \\
2: 5 $\rightarrow$ 6 $\rightarrow$ 0 $\rightarrow$ 3 \\
3: 10 $\rightarrow$ 6 $\rightarrow$ 2 \\
4: 1 $\rightarrow$ 8 \\
5: 0 $\rightarrow$ 10 $\rightarrow$ 2 \\
6: 2 $\rightarrow$ 3 $\rightarrow$ 0 \\
7: 8 $\rightarrow$ 11 \\
8: 1 $\rightarrow$ 11 $\rightarrow$ 7 $\rightarrow$ 4 \\
9: \\
10: 5 $\rightarrow$ 3 \\
11: 8 $\rightarrow$ 7 $\rightarrow$ 1
			\item Adjacency Matrix:

\begin{verbatim}
0, 0, 1, 0, 0, 1, 1, 0, 0, 0, 0, 0, 
0, 0, 0, 0, 1, 0, 0, 0, 1, 0, 0, 1, 
1, 0, 0, 1, 0, 1, 1, 0, 0, 0, 0, 0, 
0, 0, 1, 0, 0, 0, 1, 0, 0, 0, 1, 0, 
0, 1, 0, 0, 0, 0, 0, 0, 1, 0, 0, 0, 
1, 0, 1, 0, 0, 0, 0, 0, 0, 0, 1, 0, 
1, 0, 1, 1, 0, 0, 0, 0, 0, 0, 0, 0, 
0, 0, 0, 0, 0, 0, 0, 0, 1, 0, 0, 1, 
0, 1, 0, 0, 1, 0, 0, 1, 0, 0, 0, 1, 
0, 0, 0, 0, 0, 0, 0, 0, 0, 0, 0, 0, 
0, 0, 0, 1, 0, 1, 0, 0, 0, 0, 0, 0, 
0, 1, 0, 0, 0, 0, 0, 1, 1, 0, 0, 0, 
\end{verbatim}
		\end{enumerate}
	\item \todo{answer is in Ipad}
	\item \todo{answer is in Ipad}
	\item Consider by contradiction that the edge of maximum weight in the cycle C, edge e, belongs to the MST of the graph.
Since MSTs do not contain cycles there is at least one edge in C that is not in the MST. Let's call one of these edges f.
Now add f to the MST. There is now a cycle in the MST. Since e has the maximum weight in the cycle C and all edge weights are distinct, it means that weight(f) < weight(e). 
Removing the edge e after having added the edge f would generate a new MST' with total weight less than the total weight in MST, contradicting its minimality.
	\item 
		\begin{enumerate}
			\item
0: 6 $\rightarrow$ 5 \\
1: \\
2: 0 $\rightarrow$ 3 \\
3: 10 $\rightarrow$ 6 \\
4: 1 \\
5: 10 $\rightarrow$ 2 \\
6: 2 \\
7: 8 $\rightarrow$ 11 \\
8: 1 $\rightarrow$ 4 \\
9: \\
10: 3 \\
11: 8
			\item Adjacency Matrix:
\begin{verbatim}
0, 0, 0, 0, 0, 1, 1, 0, 0, 0, 0, 0, 
0, 0, 0, 0, 0, 0, 0, 0, 0, 0, 0, 0, 
1, 0, 0, 1, 0, 0, 0, 0, 0, 0, 0, 0, 
0, 0, 0, 0, 0, 0, 1, 0, 0, 0, 1, 0, 
0, 1, 0, 0, 0, 0, 0, 0, 0, 0, 0, 0, 
0, 0, 1, 0, 0, 0, 0, 0, 0, 0, 1, 0, 
0, 0, 1, 0, 0, 0, 0, 0, 0, 0, 0, 0, 
0, 0, 0, 0, 0, 0, 0, 0, 1, 0, 0, 1, 
0, 1, 0, 0, 1, 0, 0, 0, 0, 0, 0, 0, 
0, 0, 0, 0, 0, 0, 0, 0, 0, 0, 0, 0, 
0, 0, 0, 1, 0, 0, 0, 0, 0, 0, 0, 0, 
0, 0, 0, 0, 0, 0, 0, 0, 1, 0, 0, 0, 
\end{verbatim}
		\end{enumerate}
	\item \todo{Show steps}
It's strongest component is 0 2 3 5 6 10.
\end{enumerate}

\end{document}


